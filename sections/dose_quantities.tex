\section{Dose Quantities}


\subsection{Absorbed Dose}
For mass $dm$ with mean energy imparted $d\bar{\epsilon}$ by ionizing radiation
\[
D \equiv \frac{d\bar{\epsilon}}{dm} \qquad \dot{D} \equiv \frac{dD}{dt}
\]
Measured in Gray
\[
\SI{1}{\gray} = \frac{\SI{1}{\joule}}{\SI{1}{\kilo\gram}} = \SI{100}{\rad} \\
\]
For thin layer with constant stopping power,
\[D = \left( \frac{1}{\rho} \frac{dE}{dx} \right) \Phi \quad \Phi = \text{Flux} \left[ \si[per-mode=symbol]{\number\per\square\centi\meter\per\second} \right] \] 

\subsection{Quality Factor}
Q weights absorbed dose to the biological effectiveness of \textbf{charged} particles producing absorbed dose. Function of LET 

\subsection{Dose Equivalent, $H$}
\[
H \equiv Q D
\]
Used for routine radiation limits, not for acute doses \\
Measured in Sieverts
\[
\SI{1}{\sievert} = \frac{\SI{1}{\joule}}{\SI{1}{\kilo\gram}} = \SI{100}{\rem} \\
\]

\subsection{Equivalent Dose, $H_T$}
\begin{align*}
H_T & \equiv \sum_R w_R D_{T,R} \\
D_{T,R} & \equiv \text{Mean tissue dose of radiation type R} \\
w_R & \equiv \text{Radiation weighting factor}
\end{align*}
Radiation weighting factor relates type of particle or radiation for effectiveness of producing an absorbed dose

\subsubsection{Effective Dose, $E$}
\[
E \equiv \sum_T w_T H_T = \sum_T w_T \sum_R w_R D_{T,R}
\]
$w_T$ is the tissue weighting factor, accounting for relative detriment